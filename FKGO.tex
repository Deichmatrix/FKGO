\documentclass{article}
\usepackage[T1]{fontenc}                
\usepackage[utf8]{inputenc} 
\usepackage[ngerman]{babel} 
\usepackage{enumerate}
\usepackage{geometry}
\usepackage{titlesec}
\usepackage{hyperref}
\usepackage{ifthen}
\usepackage{color}
\usepackage[official]{eurosym}

\setlength\parindent{0pt}
\DeclareUnicodeCharacter{20AC}{\euro}
\geometry {a4paper, top= 25mm, bottom=25mm, left=25mm, right=25mm}
\titleformat{\part}{\fontsize{15pt}{15pt}\bfseries}{\thepart .\ }{0pt}{}
\titleformat{\chapter}{\fontsize{15pt}{15pt}\bfseries}{\thechapter .\ }{0pt}{}
\titleformat{\section}{\fontsize{11pt}{13pt}\bfseries}{\S \ \thesection \ }{0pt}{\normalsize}

\begin{document}
\noindent
\begin{center}
    \huge \textbf{Geschäftsordnung der Fachschaftenkonferenz (FKGO)}
\end{center}

\section*{Präambel}
\noindent

\part{Allgemeines}
\section{Die Fachschaftenkonferenz}
\begin{enumerate}[(1)]
    \item Vertreter aller Fachschaften der Rheinischen Friedrich-Wilhelms-Universität Bonn (RFWU Bonn) bilden die Fachschaftenkonferenz (FK), sie ist an keine Amtszeit gebunden.
    \item Die FK ist die Vollversammlung der Fachschaften der RFWU Bonn. Sie dient dem Erfahrungsaustausch der Fachschaften, beschließt Empfehlungen über alle die Fachschaften betreffenden Fragen und beschließt über die Zuweisung der Besonderen Fachschaftengelder (BFSG).
    \item Jede Fachschaft ist angehalten, einen Vertreter auf die Fachschaftenkonferenz zu entsenden und deren Protokolle zu lesen.
\end{enumerate}

\section{Das Fachschaftenkollektiv}
\begin{enumerate}[(1)]
    \item Das Fachschaftenkollektiv (FSK) ist ausführendes Organ der FK. Es besteht aus einem Vorsitzenden und mindestens 3 weiteren Mitgliedern.
    \item Der Vorsitzende führt die Geschäfte des FSK und trägt dafür die Verantwortung. Er vertritt die FK gegenüber dem Studierendenparlament (SP), den sonstigen Organen der Studierendenschaft und der Universität.
    \item Die Mitglieder des FSK unterstützen den Vorsitzenden bei seiner Arbeit. Dazu kann der Vorsitzende ihnen einzelne Aufgabenbereiche sowie seine Vertretung übertragen. Soweit die Mitglieder einen festen Aufgabenbereich haben, sind sie dem Vorsitzenden und der FK gegenüber für diesen verantwortlich.
    \item Das FSK entscheidet im Regelfall nach dem Konsensprinzip.
    \item Das FSK und seine Mitglieder sind zugleich autonomes Referat des AStA der RFWU Bonn [Fachschaftenreferat]. Der Vorsitzende des FSK ist zugleich der Referent des Fachschaftenreferates. Die übrigen Mitglieder des FSK sind die Mitarbeiter des Referats.
    \item Es müssen mindestend die Hälfte der Mitglieder des FSK auf jeder FK anwesend sein.
\end{enumerate}

\section{Protokoll}
\begin{enumerate}[(1)]
    \item Das Protokoll der FK heißt Fachschaften-Informationsdienst, kurz FID.
    \item Der Vorsitzende oder ein anderes Mitglied des FSK führt über den Verlauf der Sitzung ein Ergebnisprotokoll. Jegliche Nachbereitung und Ausarbeitung des Protokolls muss durch ein Mitglied des FSK erfolgen, welches auf der betreffenden Sitzung persönlich anwesend war.
    \item Der FID muss die wesentlichen Diskussionsinhalte und Ergebnisse einer Sitzung so wiedergeben, dass auch Personen, die nicht auf der Sitzung anwesend waren, diese nachvollziehen können. Dazu soll sich, soweit anwendbar, vollen grammatikalisch korrekten Sätzen bedient werden.
    \item Der FID muss mindestens enthalten:
    \begin{enumerate}[1.]
        \item Kandidaturen und Ergebnisse von Wahlen und Entsendungen, inklusive der vollständigen Namen und Fachschaftszugehörigkeit der betroffenen Personen.
        \item Anmerkungen und Kritiken zu Finanzanträgen
        \item Abstimmungsergebnisse zu Anträgen
        \item Den beschlossenen Wortlaut von Anträgen, welche nicht unter §§ 28 bis 31 fallen.
    \end{enumerate}
    \item Das Protokoll wird ausschließlich digital versandt. Die Fachschaften tragen dafür Sorge, dass dem Fachschaftenreferat ihre aktuelle E-Mail-Adresse bekannt ist.
    \item Das Protokoll wird nach Versand an die Fachschaften auf der nächstmöglichen Sitzung zur Genehmigung gestellt.
\end{enumerate}

\part{Gang der Verhandlung, Mitwirkungsrechte}
\section{Zusammentreten}
\begin{enumerate}[(1)]
    \item Die FK tritt an jedem Montag der Vorlesungszeit zusammen. Ausgenommen sind Feiertage,  die Montage der Weihnachtsferien sowie vom Vorsitzenden des FSK rechtzeitig bekannt gegebene weitere Termine. Die Uhrzeit wird vom Vorsitzenden der Fachschaftenkonferenz rechtzeitig bekannt gegeben.
    \item In der vorlesungsfreien Zeit finden mindestens zwei Fachschaftenkonferenzen statt [FerienFK], deren Termine vom Vorsitzenden des FSK vor Ende der Vorlesungszeit bekannt gegeben werden.
    \item Der Vorsitzende des FSK beruft auf Antrag des Ältestenrates oder von 5 Fachschaften eine Sonderfachschaftenkonferenz ein. Im Antrag ist der zu behandelnde Tagesordnungspunkt zu nennen. Der Vorsitzende des FSK kann eine Sonderfachschaftenkonferenz zudem nach eigenem Ermessen einberufen. Eine Ladungsfrist von 48 Stunden ist einzuhalten.    
\end{enumerate}

\section{Tagesordnung (TO)}
\begin{enumerate}[(1)]
    \item Die TO jeder regulären Sitzung muss folgende Punkte enthalten:
    \begin{enumerate}[1.]
        \item Berichte aus den Fachbereichen
        \item Berichte aus dem Referat
        \item Berichte aus dem AStA und den Gremien
        \item Sonstiges
    \end{enumerate}
\end{enumerate}

\section{Leitung der Sitzung}
\begin{enumerate}[(1)]
    \item Der Vorsitzende des FSK sitzt der FK vor. Er eröffnet, leitet und schließt die Sitzung (Sitzungsleitung).
    \item Der Vorsitzende des FSK wird auf eigenen Wunsch, bei Verhinderung, bei Verlassen der Sitzung oder durch Beschluss der FK von einem anderen Mitglied des FSK vertreten.
    \item Die Sitzungsleitung übt im Sitzungssaal das Hausrecht aus und wahrt die Ordnung im Sitzungsraum.
\end{enumerate}

\section{Aussprache}
\begin{enumerate}[(1)]
    \item Die Sitzungsleitung eröffnet über jeden Verhandlungsgegenstand, der auf der Tagesordnung steht, die Aussprache.
    \item Bei einer Aussprache darf nur die Person sprechen, der von der Sitzungsleitung das Wort erteilt wurde. Im Anschluss an den Wortbeitrag kann die Sitzungsleitung das Wort zu einer direkten Erwiderung oder Nachfrage erteilen, wenn sie es für zweckmäßig hält.
    \item Die Sitzungsleitung erteilt das Wort in der Reihe der Wortmeldungen, sie kann hiervon abweichen, wenn ihr dies für den Gang der Beratung dienlich erscheint.
    \item Ist die Redeliste erschöpft und meldet sich niemand zu Wort, so erklärt die Sitzungsleitung die Aussprache für geschlossen.
\end{enumerate}

\section{Äußerungen und Anträge zur Geschäftsordnung}
\begin{enumerate}[(1)]
    \item Äußerungen und Anträge zur Geschäftsordnung (GO) dürfen sich nur mit dem Gang der Verhandlung befassen.
    \item Äußerungen zur GO sind:
    \begin{enumerate}[a)]
        \item ein Hinweis zur GO;
        \item eine Anfrage zur GO;
        \item das Zurückziehen eines Antrags oder einer Anfrage.
    \end{enumerate}
    \item Anträge zur GO sind:
    \begin{enumerate}[a)]
        \item der Antrag auf Aussetzung; seine Annahme hat zur Folge, dass der Punkt nicht weiter behandelt wird und auf einer kommenden FK wieder aufgenommen werden kann;
        \item der Antrag auf Vertagung; seine Annahme hat zur Folge, dass der Punkt nicht weiter behandelt wird und auf der folgenden FK behandelt werden muss;
        \item der Antrag auf Nichtbefassung; seine Annahme hat zur Folge, dass der Punkt nicht erörtert wird;
        \item der Antrag auf Übergang zur Tagesordnung; seine Annahme hat die sofortige Behandlung des folgenden Tagesordnungspunktes oder Tagesordnungsunterpunktes zur Folge;
        \item der Antrag auf Schluss der Debatte und sofortigen Abstimmung nach vorheriger Verlesung der Redeliste;
        \item der Antrag auf Schluss der Redeliste nach vorheriger Verlesung der Redeliste und Ergänzung um weitere Wortmeldungen;
        \item der Antrag auf Beschränkung der Redezeit;
        \item der Antrag auf zeitliche Begrenzung eines Tagesordnungspunktes;
        \item der Antrag auf Beendigung der Sitzung;
        \item der Antrag auf Teilung eines Antrags in zwei oder mehrere Anträge;
        \item der Antrag auf erneute Auszählung einer Abstimmung; diesem Antrag muss auf Verlangen eines Mitglieds stattgegeben werden.Wird nach zweimaliger Auszählung kein eindeutiges Abstimmungsergebnis festgestellt, so findet die Auszählung durch namentlichen Aufruf der Fachschaftsvertreter durch die Sitzungsleitung statt. Bei einer erneuten Auszählung dürfen nur die Stimmen der Fachschaftsvertreter berücksichtigt werden, die an der Abstimmung teilgenommen haben;
        \item der Antrag auf Wiederaufnahme eines Tagesordnungspunktes;
        \item der Antrag auf Wahl einer neuen Sitzungsleitung;
        \item der Antrag auf Nichtöffentlichkeit eines Tagesordnungspunktes;
        \item Antrag auf erneute Abstimmung;
        \item Antrag auf Vertagung bis zum Erhalt zusätzlicher Informationen, diesem Antrag muss auf Verlangen von mindestens 3 Fachschaften stattgegeben werden. Das Verlangen muss explizit im FID wiedergegeben werden;
        \item Antrag auf Änderung der Tagesordnung.
    \end{enumerate}
    \item Zu einer Meldung zur GO erteilt die Sitzungsleitung das Wort unmittelbar und außerhalb der Redeliste; ein laufender Redebeitrag darf nicht unterbrochen werden. Meldungen zur GO werden durch das Heben beider Hände angezeigt.
    \item Die Worterteilung ist bei Anträgen, denen entsprochen werden muss (Verlangen), auf den Antragssteller zu beschränken.
    \item Erhebt sich zu einem GO-Antrag kein Widerspruch, so gilt er als angenommen; andernfalls ist der Antrag nach Anhörung einer Gegenrede abzustimmen.
\end{enumerate}

\section{Auskunftspflicht}
\begin{enumerate}[(1)]
    \item Auf begründetes Verlangen von mehr als 2 Fachschaften hat ein Vertreter einer bestimmten Fachschaft auf der nachfolgenden Sitzung anwesend zu sein und Auskunft zu erteilen (Zitierrecht).
    \item Auf begründetes Verlangen von mehr als 2 Fachschaften hat ein von der FK entsendeter Vertreter eines Ausschusses auf der nachfolgenden Sitzung anwesend zu sein und Auskunft zu erteilen.
    \item Auf begründetes Verlangen von mehr als 2 Fachschaften hat ein bestimmtes FSK Mitglied auf der nachfolgenden Sitzung anwesend zu sein und Auskunft zu erteilen.
    \item Der Vorsitzende des FSK hat die zitierte Fachschaft oder Person hierüber frühzeitig zu informieren. 
    \item Mitglieder der Fachschaften und des FSK haben vor Abstimmungen offenzulegen, ob sie durch Annahme persönlich oder als Mitglied einer Organisation profitieren (Befangenheit). Dies gilt insbesondere für Beschlüsse zu Beitragsordnungen, Haushaltsplänen und der Satzung der Studierendenschaft.
\end{enumerate}

\section{Stimmrecht}
\begin{enumerate}[(1)]
    \item Stimmberechtigt sind die von einem Organ einer Fachschaft dazu bevollmächtigten Vertreter (Delegierte der Fachschaften). Jede Fachschaft hat eine Stimme.
    \item Die Delegierten der Fachschaften haben ihre Bevollmächtigung auf Verlangen des FSK oder von mehr als 2 Fachschaften nachzuweisen.
    \item Mitglieder des FSK können nicht Delegierte einer Fachschaft sein.
\end{enumerate}

\section{Antragsrecht}
Antragsberechtigt sind alle Studierenden der RFWU Bonn, die Delegierten der Fachschaften sowie die Vertreter der nach der Satzung der Studierendenschaft ordnungsgemäß gewählten Organe der Fachschaften und das FSK.

\section{Rederecht}
Jedes Mitglied der Studierendenschaft der RFWU Bonn hat Rederecht.

\section{Öffentlichkeit}
\begin{enumerate}[(1)]
    \item Die Sitzungen sind öffentlich.
    \item Auf Antrag kann die FK die Öffentlichkeit zu einzelnen Tagesordnungspunkten ausschließen. Über den Antrag wird in nichtöffentlicher Sitzung beraten und beschlossen.
    \item Zur Öffentlichkeit gehört nicht der Ältestenrat. Einzelne Personen können von der FK zur Beratung hinzugezogen werden.
\end{enumerate}

\part{Beschlussfassung}
\section{Beschlussfähigkeit}
\begin{enumerate}[(1)]
    \item Die FK ist beschlussfähig, wenn mindestens 1/5 aller Fachschaften vertreten sind.
    \item Die Beschlussfähigkeit wird auf Antrag unverzüglich durch die Sitzungsleitung festgestellt.
    \item Die FK gilt solange als beschlussfähig, bis das Fehlen der Beschlussfähigkeit festgestellt worden ist.
    \item Abs. 1-3 finden auch auf Ferien- und Sonderfachschaftenkonferenzen Anwendung.
\end{enumerate}

\section{Lesungen}
\begin{enumerate}[(1)]
    \item Anträge auf Beschlussfassung werden grundsätzlich in 3 Lesungen behandelt.
    \item Der Abstand zwischen der ersten und zweiten Lesung darf 24 Stunden nicht unter- und 30 Tage nicht überschreiten. Der Abstand zwischen der zweiten und dritten Lesung darf 30 Tage nicht überschreiten.
    \item In der ersten Lesung wird der Antrag vorgestellt und begründet. In der zweiten Lesung wird der Antrag debattiert und Änderungsanträge werden eingebracht und abgestimmt. In der dritten Lesung wird über den Antrag unter Berücksichtgung der angenommenen Änderungsanträge abgestimmt.
    \item In Sonderfällen kann die FK einstimmig beschließen alle drei Lesungen auf einer Sitzung zu behandeln. Ausgenommen davon sind Anträge und Stellungnahmen zu Finanzen, Satzungen, Ordnungen, Statuten, sowie Personalwahlen.
\end{enumerate}

\section{Abstimmungen}
\begin{enumerate}[(1)]
    \item Abstimmungen werden grundsätzlich durch Handzeichen durchgeführt. Auf Verlangen eines Delegierten oder FSK Mitglieds ist die Abstimmung geheim durchzuführen. Über GO-Anträge, die Überweisung einer Sache an einen Ausschuss und den Antrag auf Änderung der Tagesordnung kann nicht geheim abgestimmt werden.
    \item Liegen mehrere Anträge zur selben Sache vor, so wird über den weitestgehenden zuerst abgestimmt. Bei Finanzanträgen ist über den am wenigsten weitgehenden Antrag zuerst abzustimmen.
    \item Das Ergebnis der Abstimmung wird durch die Sitzungsleitung festgestellt.
    \item Ein Beschluss ist rechtmäßig zustande gekommen, wenn
    \begin{enumerate}[1.]
        \item die FK beschlussfähig war und
        \item er die einfache Mehrheit gefunden hat, soweit diese GO nichts anderes vorschreibt.
    \end{enumerate}
    \item Wenn sowohl die Zahl der Ja-Stimmen als auch die Zahl der Nein-Stimmen geringer ist als die Zahl der Enthaltungen, kann auf Antrag eines Mitglieds des FSK oder eines stimmberechtigten Mitglieds der FK der Antrag einmalig vertagt werden. Die Abstimmung ist in diesem Fall ergebnislos.
\end{enumerate}

\part{Das Fachschaftenkollektiv (FSK)}
\section{Wahl des Vorsitzenden und der weiteren Mitglieder des FSK}
\begin{enumerate}[(1)]
    \item Der Vorsitzende und die weiteren Mitglieder des FSK werden zu Beginn des Wintersemesters mit einfacher Mehrheit für ein Jahr gewählt. Sie können zurücktreten oder auf Antrag von mindestens fünf Fachschaften abgewählt werden. Die Abwahl oder der Rücktritt des Vorsitzenden ist nur unter gleichzeitiger Neuwahl des Amtes möglich. § 15 gilt entsprechend.
    \item Zusätzliche Mitglieder des FSK können jederzeit gewählt werden. Ihre Amtszeit reicht bis zum Beginn des nächsten Wintersemesters.
    \item Die Kandidaten haben sich auf Verlangen vorzustellen.
    \item Das Amt des Vorsitzenden des FSK ist unvereinbar mit der Mitgliedschaft in einer FSV, der Mitgliedschaft in einem FSR oder einem weiteren Amt innerhalb des AStA.
    \item Das Amt eines weiteren Mitgliedes des FSK ist unvereinbar mit der Mitgliedschaft in einem FSV-Vorstand, der Mitgliedschaft in einem FSR-Vorstand, dem Posten des Finanzreferenten einer Fachschaft oder einem Referentenposten innerhalb des AStA.
    \item Zur Durchführung besonderer Aufgaben kann das FR auf Beschluss der Fachschaftenkonferenz studentische Mitarbeiterinnen zeitlich begrenzt beschäftigen. Aushilfsstellen sind öffentlich auszuschreiben. Sie erhalten eine angemessene Aufwandsentschädigung. Sie sind nicht Teil des FSK und dürfen keine Finanzangelegenheiten bearbeiten.
    \item Die Wahl ist geheim. 
\end{enumerate}

\section{Aufgaben des FSK}
\begin{enumerate}[(1)]
    \item  Die Aufgaben des FSK sind:
    \begin{enumerate}[1.]
        \item die Vor- und Nachbereitung von Sitzungen der FK;
        \item die Sichtung und Aufbereitung von Anträgen auf BFsG;
        \item die Prüfung von Anträgen auf AFsG;
        \item die Bereitstellung von Materialien für die Durchführung von Fachschaftswahlen im Sinne der Fachschaftswahlordnung (FSWO);
        \item die Vertretung der Fachschaftenkonferenz gegenüber dem Allgemeinen Studierendenausschuss (AStA), dem Studierendenparlament (SP), der Universitätsverwaltung, dem Studierendenwerk und dem Rektorat der Universität.
    \end{enumerate}
    \item Darüber hinaus ist das FSK für die folgenden Aufgaben von gehobener Bedeutung für Fachschaften verantwortlich:
    \begin{enumerate}[1.]
        \item die Sicherung der ordnungsgemäßen und demokratischen Arbeit der Fachschaften im Sinne der Satzung der Studierendenschaft (SdS) und dieser Geschäftsordnung;
        \item Die Vertretung der Interessen der Fachschaften im Bezug auf Rechtsakte der Studierendenschaft, insbesondere Änderungen der Satzung der Studierendenschaft (SdS) sowie Beschlüsse von Beitragsordnungen und Haushaltsplänen;
        \item Aktualisierung der Fachschaftenliste (Anhang III FKGO), diese ist auf der ersten FK eines neuen Semesters vorzulegen.
    \end{enumerate}
    \item Die Verantwortung für die Erfüllung der Aufgaben mit gehobener Bedeutung liegt beim Vorsitzenden des FSK. Er hat die Arbeit der Mitarbeiter des FSK so zuzuteilen, dass die in Abs. 2 genannten Aufgaben erfüllt werden. Dabei ist, falls notwendig, Vorrang gegenüber anderer Aufgaben, insbesondere der Bearbeitung von Finanzanträgen zu nehmen.
    \item Der Vorsitzende ist für die Anweisung der Auszahlung von Geldern an die Fachschaften im Sinne dieser Geschäftsordnung verantwortlich. Diese Aufgabe ist nicht übertragbar. Sind die Bedingungen dieser Geschäftsordnung nicht erfüllt, muss er die Anweisung verweigern und die FK auf vorliegende Mängel hinweisen.
    \item Das FSK wirkt unbeschadet der Aufsichtsrechte des Rektorates darauf hin, dass die Organe der Fachschaften ihre Aufgaben und Pflichten im Rahmen der gesetzlichen Bestimmungen, der HWVO und dieser Satzung erfüllen und teilt Unterlassungen oder Verstöße der FK mit. Hält das FSK Beschlüsse, Haushaltsführung, Maßnahmen oder Unterlassungen der Fachschaften für rechtswidrig, so kann sie mit der Mehrheit seiner Mitglieder Abhilfe verlangen. Dies ist der FK zu berichten. 
    Sollte durch die betroffene Fachschaft innerhalb einer angemessenen Frist keine Abhilfe geschaffen werden, so hat der Vorsitz des FSK das Rektorat zu informieren. Der Vorsitz des FSK hat das Recht und auf Antrag des Vorsitzes eines Fachschaftsrates die Pflicht das Rektorat unverzüglich zu informieren.
\end{enumerate}

\part{Ordnungsmaßnahmen}
\section{Sach- und Ordnungsruf}
\begin{enumerate}[(1)]
    \item Die Sitzungsleitung kann einen Redner, der vom Verhandlungsgegenstand abweicht, zur Sache verweisen [Sachruf]. Er kann Anwesende, wenn sie die Ordnung verletzen, zur Ordnung rufen [Ordnungsruf]. Der Ordnungsruf und der Anlass hierzu dürfen von den nachfolgenden Rednern nicht thematisiert werden.
    \item Gegen einen solchen Ordnungsruf kann der Betroffene nur unverzüglich Einspruch einlegen.
    \item Über den Einspruch entscheidet die FK mit einfacher Mehrheit.
    \item Gegen einen Ordnungsruf ist ein Widerspruch vor dem Ältestenrat möglich, er hat keine aufschiebende Wirkung.
\end{enumerate}

\section{Wortentziehung}
Ist ein Redner während einer Rede dreimal zur Sache gerufen worden und beim zweiten Mal auf die Folgen eines dritten Rufes hingewiesen worden, so entzieht die Sitzungsleitung ihm das Wort.

\section{Ausschluss von der Sitzung}
Wurde eine Person während einer Sitzung dreimal zur Ordnung gerufen und beim zweiten Mal auf die Folgen eins dritten Rufes hingewiesen, so schließt die Sitzungsleitung sie von der Sitzung aus.

\section{Unterbrechung der Sitzung}
Wenn im Sitzungsraum störende Unruhe entsteht, die den Fortgang der Verhandlungen in Frage stellt, kann die Sitzungsleitung die Sitzung auf unbestimmte Zeit unterbrechen. Die Fortsetzung der Sitzung erklärt die Sitzungsleitung nach eigenem Ermessen.

\part{Ausschüsse}
\section{Ständige Ausschüsse}
\begin{enumerate}[(1)]
    \item Die ständigen Ausschüsse der FK bilden:
    \begin{enumerate}
        \item der Wahlprüfungsausschuss der Fachschaften (WPAF)
        \item der Geschäftsordnungs- und Satzungsausschuss der Fachschaftenkonferenz (GoSaFK)
        \item der Haushaltsausschuss der Fachschaften (HauF)
    \end{enumerate}
    \item Die ständigen Ausschüsse der FK bestehen aus mindestens 3 durch die FK zum Beginn des Sommersemester gewählten Mitgliedern.
    \item Mitglieder von Ausschüssen werden mit einfacher Mehrheit gewählt. Die Kandidaten haben sich auf Verlangen vorzustellen. Es gilt § 15.
    \item Die Ausschussmitglieder müssen Studierende der RFWU Bonn sein. Ausschüsse sollen geschlechter- und fakultätsparitätisch besetzt werden.
    \item Die Ausschüsse bestimmen auf der FK aus ihrer Mitte einen Vorsitzenden. Der Ausschussvorsitzende beruft die Sitzungen ein, leitet diese und ist für die Weiterleitung von Arbeitsergebnissen verantwortlich.
    \item Die Bestimmungen zum Wahlprüfungsausschuss der Fachschaften (WPAF) richten sich nach § 22 der Fachschaftswahlordnung (FSWO).
    \item Der Satzungs- und Geschäftsordnungsausschuss (GoSaFK) ist für die Ausarbeitung von Dokumenten, welche die Arbeit der Fachschaften regeln, zuständig. Dazu gehören insbesondere diese Geschäftsordnung und die Wahlordnung für die Wahlen der Fachschaftsvertretungen und Fachschaftsräte (FSWO). 
    \item Der Haushaltsausschuss (HauF) ist in Abstimmung mit FSK und AStA Finanzreferat zuständig für die Planung der Selbstbewirtschaftungsmittel der Fachschaften in Anlage 1 zur FKGO, Beitragsordnung (BO) und im Haushaltsplan (HHP) der Studierendenschaft. Der HauF erarbeitet die Beschlussempfehlungen der FK bezüglich dem HHP und der BO; er ist durch das FSK hierfür mindestens 14 Tage vorher anzurufen und mit den nötigen Informationen zur Haushaltführung der Fachschaften zu versorgen. Vorsitz des HauF muss ein Mitglied des FSK sein.
\end{enumerate}

\section{Sonstige Ausschüsse und Gremien}
\begin{enumerate}[(1)]
    \item Die FK kann durch Mehrheitsbeschluss weitere Ausschüsse bilden. Die sonstigen Ausschüsse der FK haben beratende Funktion und erarbeiten Empfehlungen an die FK. 
    \item Für alle Gremien gelten § 23 Abs. 3 bis 5 entsprechend.
    \item Die Fachschaften einer Fakultät haben das Recht, eine Fakultäts-Fachschaftenkonferenz zu bilden und über diese Vertretungen für die Gremien der Fakultät zu nominieren. 
    \item Die Fachschaften können zur Arbeit an Themenbereichen, die über die Kapazität der FK hinaus gehen, themenbezogene Teilkonferenzen bilden. 
    \item Zu den Sitzungen wird mit mindestens einer Woche Vorlauf durch den Vorsitzenden des Gremiums eingeladen, sofern das Gremium in einer eigenen Geschäftsordnung keine andere Regelung trifft. Die Einladung ist zusammen mit der vorläufigen Tagesordnung zu verschicken. § 4 findet keine Anwendung.
    \item Die Bildung solcher Gremien ist der FK anzuzeigen. Sofern sich die Gremien keine eigene Geschäftsordnung geben, gilt die FKGO. Nach jeder Sitzung ist auf der nachfolgenden FK ist Bericht zu erstatten.
    \item Die FK entsendet gemäß § 12 Abs. 7 SdS je ein ordentliches und ein stellvertretendes Mitglied in die Ausschüsse des Studierendenparlaments. Der Vorsitzende des FSK weist die Fachschaften auf unbesetzte Ausschüsse hin, leitet die Nominierungen der FK umgehend an das Präsidium des Studierendenparlaments weiter und stellt eine regelmäßige Berichterstattung über die Ausschussarbeit sicher.
\end{enumerate}

\part{Finanzen}
\section{Allgemeine Bestimmungen}
\begin{enumerate}[(1)]
    \item Die den Fachschaften gemäß § 43 Satzung der Studierendenschaft der RFWU Bonn zugewiesenen Gelder werden durch das FSK verwaltet.
    \item Die Gelder werden den einzelnen Fachschaften als Allgemeine Fachschaftengelder [AFsG] und als Besondere Fachschaftengelder [BFsG] zugewiesen. AFsG setzen sich zusammen aus einem Sockelbetrag und einem weiteren Betrag, der sich nach der Zahl der Studierenden richtet, die der betreffenden Fachschaft gemäß § 22 Satzung der Studierendenschaft der RFWU Bonn zugeordnet sind. Mit BFsG werden wichtige Fachschaftsbelange gesondert gefördert.
    \item Die Gelder werden nur auf Antrag ausgezahlt.
    \item Die ersten beiden FKs eines Monats sind Finanzfachschaftenkonferenzen, soweit sie innerhalb der Vorlesungszeit liegen [Finanz-FK]. Die FK kann abweichende Termine festlegen.
    \item Grundlegende Voraussetzung für die Auszahlung von Geldern an eine Fachschaft sind:
    \begin{enumerate}[1.]
        \item die demokratische Wahl der Gremien der Fachschaft im Sinne der Fachschaftswahlordnung (FSWO) innerhalb der letzten 12 Monate vor dem Antragszeitpunkt.
        \item die vom Wahlleiter unterschriebenen Wahlergebnisse der Wahlen der Fachschaftsvertretungen, die im Antragszeitraum im Amt waren,
        \item die von Wahlleiter, Vorsitzendem und Protokollanten unterschriebenen Protokolle der konstituierenden Sitzungen der Fachschaftsvertretungen, die im Antragszeitraum im Amt waren,
        \item ein Bankkonto, welches auf die Fachschaft selbst eingetragen ist. Geld wird ausschließlich auf diese Fachschaftskonten ausgezahlt.
        \item eine gültige und im Sinne der Satzung der Studierendenschaft veröffentlichte Fachschaftssatzung. 
        \item für das aktuelle und vergangene Haushaltsjahr gültige Haushaltspläne
        \item das vom Protokollanten unterschriebene Protokoll der Sitzung, auf der dieser HHP beschlossen wurde, mit allen weiteren vom jeweiligen Protokollanten unterschriebenen Protokollen der Sitzungen, auf denen Änderungen des HHP beschlossen wurden.
        \item die Kassenabrechnungen, des vergangenen Haushaltsjahres, orientiert an den Posten des HHP mit Kassenständen zu Beginn und zu Ende des Antragszeitraums, unterschrieben vom Finanzreferenten.
        \item die von den Kassenprüfern unterschriebenen Kassenprüfungsberichte aller Kassenprüfungen, welche das vergangene Haushaltsjahr abdecken, sowie das Protokoll der Wahl der Kassenprüfer.
        \item ein von der Fachschaftsvertretung oder einer Fachschaftsvollversammlung gefasster und durch deren Vorsitzenden unterschriebener Beschluss über die finanzielle  Entlastung des Fachschaftsrates des vergangenen Haushaltsjahres.
    \end{enumerate}
    \item Kassenprüfer kann nur sein, wer weder im geprüfen Zeitraum noch zum Prüfungszeitpunkt Mitglied des FSR war beziehungsweise ist.
    \item Wurde für eine Amtsperiode keine Fachschaftsvertretung gewählt, tritt in diesen Bestimmungen an ihre Stelle die Fachschaftsvollversammlung.
    \item Die Gelder der Fachschaften dürfen nicht dazu verwendet werden, Aufgaben der Institute oder Seminare der Universität zu finanzieren. Sie dürfen außerdem nicht für Veranstaltungen verwendet werden, die für das bestehen einer Lehrveranstaltung notwendig sind oder eine solche Veranstaltung ersetzen.
    \item Antragsberechtigt sind die Fachschaftsräte der Fachschaften der RFWU Bonn.
    \item Im Falle einer Wahlprüfung ist die Anweisung bzw. Auszahlung von AFsG und BFsG an die betreffende Fachschaft auszusetzen, bis die Wahlprüfung beendet ist. Solange eine Wahlprüfung aufgrund fehlender Unterlagen nicht möglich ist, kann die betreffende Fachschaft für die betroffene Wahlperiode keine Anträge auf AFsG und BFsG stellen, außer sie hat diesen Umstand nachweislich nicht selbst zu verschulden.
    \item Die Finanzanträge einer Fachschaft dürfen im FSK nicht von einer Person mit gleicher Fachschaftszugehörigkeit geprüft werden.

\end{enumerate}

\section{AFsG}
\begin{enumerate}[(1)]
    \item Die AFsG dienen dem Bestreiten des allgemeinen Geschäftsbetriebs einer Fachschaft.
    \item Die AFsG, die eine Fachschaft zugewiesen bekommt, setzen sich zusammen aus einem Sockelbetrag und einem Betrag, der sich nach der Zahl der Studierenden richtet, die der betreffenden Fachschaft gemäß § 22 Satzung der Studierendenschaft der RFWU Bonn zugeordnet sind. Dieser Betrag wird anteilig nach den allen Fachschaften laut Haushaltsplan insgesamt zustehenden AFsG berechnet. Berechnungszeitraum ist ein Semester.
    \item Für die Studierendenzahl gilt die Auskunft der Universitätsverwaltung für das betreffende Semester.
    \item Die Gelder werden rückwirkend für die letzten Semester ausgezahlt. Der Antrag kann frühestens für das aktuell laufende Semester gestellt werden und spätestens 2 Semester nach Ablauf des Semesters, auf welches sich der Antrag bezieht.
    \item Dem FSK müssen die in §25 Abs.5 genannten Dokumente für den Antragszeitraum vorliegen.
    \item Zudem hat die beantragende Fachschaft ihre aktuellen Kontaktdaten vollständig beizulegen.
    \item Sind die Voraussetzungen nach Abs. 5 erfüllt, der Antrag ordnungsgemäß ausgefüllt und bestehen sonst keine Bedenken gegen die Auszahlung der Gelder, so werden die Gelder vom Vorsitzenden des FSK oder seinem Vertreter angewiesen.
    \item Anträge, die vier Semester nach Ablauf des Semesters, auf das sich der Antrag bezieht, die Voraussetzungen des Abs. 5 nicht erfüllen, gelten als nicht gestellt.
\end{enumerate}

\section{Sockelsatz der AFsG}
\begin{enumerate}[(1)]
    \item Der Sockelsatz der AFsG sowie anfallende Änderungen werden in Anlage I zur FKGO festgelegt.
    \item In den Beschlussempfehlungen der FK für die Beitragsordnung und der die Fachschaften betreffenden Punkte des Haushaltsplanes sind ausreichende Gelder einzuplanen, um die Auszahlung der AFsGs an alle Fachschaften für das laufende Haushaltsjahr und die zwei vorrausgegangenen Haushaltsjahre zu ermöglichen.
    \item Die AFsG werden im Haushaltsplan unter den Posten “aktuelle AFsG” und “rückwirkende AFsG” festgehalten. Dabei umfasst der Posten “aktuelle” alle Anträge die noch zum Zeitpunkt der Auszahlung noch nach §26 (4) stell bar sind. Der Posten “rückwirkend” umfasst alle ältere noch offene Anträge wobei §26 (8) gilt.
\end{enumerate}

\section{BFsG}
\begin{enumerate}[(1)]
    \item BFsG dienen der Förderung wichtiger Fachschaftsbelange. Dies sind insbesondere Fahrten zu überregionalen Fachschaftentreffen und andere Veranstaltungen, welche der Information bzw. dem Erfahrungsaustausch der Fachschaften dienen, die Ausrichtung von Erstsemesterorientierungseinheiten, Fachschaftsarbeitswochenenden und Arbeitskreisen zu bestimmten Themen, Vorträgen, Seminaren, Podiumsdiskussionen, Ausstellungen, Ringvorlesungen und ähnliches.
    \item BFsG werden auf Antrag ausgezahlt. Der Antrag ist dem FSK zur Prüfung vorzulegen. Das FSK bringt den Antrag sodann auf der ersten Finanz-FK des nächsten Monats zur Abstimmung durch die FK ein. § 15 gilt entsprechend.
    \item Die Fachschaft, die den Antrag stellt, muss auf den Finanz-FK, auf welchen ihr Antrag behandelt wird, durch einen Delegierten vertreten sein. Der Antrag ist von dem Delegierten gegebenenfalls zu begründen und zu erläutern. Ist kein Delegierter anwesend wird der Antrag in erster Lesung auf den nächsten Monat verschoben, § 15 gilt entsprechend.
    \item Der Antrag ist spätestens 2 Wochen vor der Finanz-FK, auf welcher er vorgestellt werden soll, zur Prüfung vorzulegen. 
    \item In dem Antrag sind alle entstandenen Einnahmen und Ausgaben aufzulisten. Die zu erstattenden  Kosten sind durch Kopien der Rechnungen vollständig zu belegen. Bei Fahrten  sind dem Antrag zudem eine Teilnehmerliste mit Unterschriften aller Teilnehmer hinzuzufügen.
    \item Die FK hat das Recht, Anträge zurückzuweisen, wenn sie der Ansicht ist, dass diese unberechtigt sind. Ebenso kann die FK lediglich einen Teil der beantragten Summe bewilligen.
    \item Ein abgelehnter oder teilweise bewilligter Antrag kann nicht noch einmal gestellt werden. Ein Antrag gilt als abgelehnt, wenn die FK ihn durch Mehrheitsbeschluss nicht bewilligt. Er gilt als teilweise bewilligt, wenn der Antragssumme nicht in voller Höhe zugestimmt wurde.
    \item Anträge, die sechs Monate nach Einreichung nicht vorgestellt oder abgestimmt wurden, gelten als nicht gestellt und müssen neu beantragt werden.
    \item Gelder können maximal 1 Semester rückwirkend beantragt werden.
    \item Die genaue Höhe und Ausgestaltung der BFsG bestimmt sich nach § 30 und Anlage II.
\end{enumerate}

\section{Fachschaftsübergreifende Ausgaben}
\begin{enumerate}[(1)]
    \item Das FSK kann auf der FK BFsG für fachschaftsübergreifende Maßnahmen und Anschaffungen beantragen. In diesen Fällen muss die Fachschaftenkonferenz die Anträge vor Beginn der Maßnahme entscheiden. Maßnahmen, deren zu beantragende Kosten nicht durch § 30 abgedeckt sind, und Fahrten ins Ausland bedürfen abweichend von § 31 Abs. 1 keiner Vorankündigung. § 15 gilt entsprechend.
    \item Mindestens zwei Fachschaften können auf der FK BFsG für fachschaftsübergreifende Maßnahmen und Anschaffungen beantragen. Bei Bewilligung wird die Summe auf Basis des Antrags auf die Fachschaftskonten ausgezahlt.
\end{enumerate}

\section{Kriterien zu Vergabe der BFsG}
\begin{enumerate}[(1)]
    \item Als Voraussetzung für die Beantragung von BFsG gilt § 25. Ausnahme bildet die Bezuschussung von neugegründeten Fachschaften, deren Voraussetzungen in Anlage II geregelt werden.
    \item Es werden nur für die in Anlage II definierten Kostenpunkte Anträge gewährt.
    \item Die Höchstsätze werden ebenfalls in Anlage 2 definiert.
    \item Dem Abschluss von Verträgen über Lieferungen und Leistungen muss ein Preisvergleich vorausgehen. Bei Aufträgen mit einem Wert von mehr als 1.000 Euro sind mindestens 3 Angebote im Wettbewerb einzuholen, bei Aufträgen mit einem Wert von mehr als 10.000 Euro sind mindestens 6 Bewerber/innen zur Angebotsabgabe aufzufordern. Der Preisvergleich ist aktenkundig zu machen und die Vergabeentscheidung zu dokumentieren. (vgl. § 2 Abs. 2 HWVO)
    \item Fahrten und Exkursionen ins Ausland bedürfen einer Vorankündigung.
    \item Für die Erstattung von Fahrtkosten gilt:
    \begin{enumerate}[1.]
        \item bei Fahrten mit der Bahn/ÖPNV soll der sinnvollste und der kostengünstigste Tarif gewählt werden (dass es sich um den kostengünstigsten Tarif handelt, ist gesondert zu dokumentieren) Abweichungen sind zu begründen. Fahrzeiten zwischen 8 Uhr und 22 Uhr werden als zumutbar bzw. sinnvoll erachtet. Kosten für eine Bahncard werden nur dann erstattet, wenn die Gesamtkosten der Fahrt incl. Bahncard die Gesamtkosten ohne Bahncard nicht übersteigen.
        \item Bei Anfahrt mit dem Auto werden die Kosten von maximal 0,30 €/km und Auto pauschal erstattet. Eine Erstattung von Tankkosten und Gebühren für Mietwägen erfolgt innerhalb des Maximalbetrages der Pauschale.   Es wird vorausgesetzt, dass die Kapazität der Fahrzeuge, abhängig von der Anzahl der Mitfahrenden Personen, sinnvoll ausgenutzt wird. 
        \item Bahnfahrten sind Fahrten mit dem Auto vorzuziehen. Fahrten mit dem Auto müssen begründet werden.
        \item Flüge innerhalb Deutschlands werden generell nicht erstattet.   
    \end{enumerate}
    \item Beim Druck von Zeitschriften (Ersti-Heft, Newsletter, Flyer o.ä.) sind dem Antrag ein Belegexemplar hinzuzufügen. Das Belegexemplar kann digital vorgelegt werden.
    \item Verpflegungskosten werden nicht erstattet. Die einzige Ausnahme bildet ein gemeinsames Essen im Rahmen der Erstsemesterarbeit. 
    \item Alkoholische Getränke werden grundsätzlich nicht erstattet.
    \item Kosten für die Durchführung von Lehrveranstaltungen welche von der RFWU Bonn durchzuführen sind, sind nicht ersatzfähig.
    \item Die Kosten  werden nur insoweit übernommen, wie sie nicht durch andere Einnahmen gedeckt sind.
\end{enumerate}

\section{Ausnahmegenehmigung von BFsG}
\begin{enumerate}[(1)]
    \item Maßnahmen, deren zu beantragende Kosten nicht durch § 30 abgedeckt sind, und Fahrten ins Ausland müssen vor Beginn der Maßnahme vorangekündigt werden.
    \item Vorankündigungen sind auf dem Formblatt zu stellen. Eine Begründung der Überschreitung des § 30 bzw. der Notwendigkeit, ins Ausland zu reisen, ebenso wie eine Vorkalkulation der Maßnahme, sind beizufügen.
    \item Vorankündigungen für Veranstaltungen müssen schriftlich mindestens vier Wochen vor der Veranstaltung bzw. Fahrt dem FSK vorliegen und mindestens zwei Wochen vor der Veranstaltung auf der FK vorgestellt werden.
    \item Die FK kann durch mehrheitliche Annahme der Vorankündigung beschließen, dass die jeweilige Fachschaft bei einem späteren Antrag auf Zubilligung Besonderer Fachschaftengelder in dem durch die Vorankündigung begrenzten Rahmen von § 30 abweichen kann beziehungsweise die Erstattung der Kosten für eine Fahrt ins Ausland beantragen darf. Die Abstimmung der Vorankündigung erfolgt in der auf die Vorstellung folgenden FK.
    \item Die FK hat das Recht, Anträge zurückzuweisen, wenn sie der Ansicht ist, dass diese unberechtigt sind. Ebenso kann die FK lediglich einen Teil der beantragten Summe bewilligen.
    \item Bei der Vorstellung und Abstimmung von Vorankündigungen muss ein Delegierter der antragstellenden Fachschaft anwesend sein, um den Antrag zu begründen und gestellte Fragen zu beantworten. Ist die antragstellende Fachschaft nicht anwesend, müssen die Anträge zurückgestellt werden.
    \item Beschließt die FK die Vorankündigung nicht, gilt die Maßnahme als nicht vorangekündigt. Eine Abweichung von § 30 bzw. die Erstattung der Kosten für eine Fahrt ins Ausland ist dann nicht möglich.
    \item Vorankündigungen über Anschaffungen müssen vor dem Kauf beschlossen werden. Die Anschaffung kann sofort nach dem Beschluss gekauft werden. Eine Frist von 4 Wochen ist nicht einzuhalten.
    \item BFSG Anträge, deren Kosten in mehreren Punkten nicht durch §30 abgedeckt sind, benötigen nur einer Vorankündigung, in der alle Überschreitungen genannt werden.
    \item Das Erstatten alkoholischer Verpflegung ist auch durch Vorankündigungen nicht möglich.
    \item Veranstaltungen, die durch Anlage II nicht als Erstiarbeit oder inhaltliche Veranstaltung definiert sind,  bleiben auch durch eine Vorankündigung nicht förderbar.
    
\end{enumerate}
 
\section{Kaution für Veranstaltungen}
\begin{enumerate}[(1)]
    \item Fachschaften können für Veranstaltungen einen Vorschuss für an das Studierendenwerk oder die Universität zu leistende Kautionen beantragen. Anträge dazu sind an das Fachschaftenkollektiv zu richten und werden von diesem beschieden. Dem Antrag ist eine Vorkalkulation der Veranstaltung beizufügen.
    \item Die geleisteten Vorschüsse sind unmittelbar nach der Veranstaltung, spätestens bis zum 14. Tag nach der Veranstaltung, auf das Konto des AStA zurückzuleisten.
    \item Die geleisteten Vorschüsse können der betreffenden Fachschaft gegenüber mit noch auszuzahlenden AFsG und BFsG aufgerechnet werden.
\end{enumerate}

\section{Finanzleitfaden}
Das FSK stellt den Fachschaften einen aktuellen Auszug der Geschäftsordnung und der Anlage II als Leitfaden über die Regelungen zur Verteilung und Beantragung der für die Fachschaften eingezogenen Beträge zur Verfügung.

\part{Schlussbestimmungen}
\section{Schlussbestimmungen}
\begin{enumerate}[(1)]
    \item Diese Geschäftsordnung tritt mit ihrer Veröffentlichung in der AKUT in Kraft.
    \item Änderungen der Geschäftsordnung bedürfen einer Zweidrittelmehrheit einer FK, auf der mindestens 20\% der Fachschaften vertreten sind. § 15 gilt entsprechend.
    \item Im Falle einer unplanmäßigen Regelungslücke ist die Geschäftsordnung des Studierendenparlaments der RFWU Bonn entsprechend anzuwenden.   
\end{enumerate}

\section{Übergangsbestimmungen}
Abweichend von § 26 Abs. 8 gelten Anträge auf AFsG, die vor dem Sommersemester 2020 gestellt wurden, ab dem 01. Oktober 2021 als nicht gestellt.
\end{document}