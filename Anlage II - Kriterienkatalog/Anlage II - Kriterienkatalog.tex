\documentclass[11pt]{article}
\usepackage[T1]{fontenc}                
\usepackage[utf8]{inputenc} 
\usepackage[ngerman]{babel} 
\usepackage{enumerate}
\usepackage{geometry}
\usepackage{titlesec}
\usepackage{hyperref}
\usepackage{ifthen}
\usepackage{color}
\usepackage{tabularx}
\usepackage[official]{eurosym}

\setlength\parindent{0pt}
\DeclareUnicodeCharacter{20AC}{\euro}
\geometry {a4paper, top= 25mm, bottom=25mm, left=25mm, right=25mm}
\titleformat{\part}{\fontsize{15pt}{15pt}\bfseries}{\thepart .\ }{0pt}{}
\titleformat{\chapter}{\fontsize{15pt}{15pt}\bfseries}{\thechapter .\ }{0pt}{}
\titleformat{\section}{\fontsize{11pt}{13pt}\bfseries}{\S \ \thesection \ }{0pt}{\normalsize}

\newcolumntype{L}[1]{>{\raggedright\arraybackslash}p{#1}} % linksbündig mit Breitenangabe
\newcolumntype{C}[1]{>{\centering\arraybackslash}p{#1}} % zentriert mit Breitenangabe
\newcolumntype{R}[1]{>{\raggedleft\arraybackslash}p{#1}} % rechtsbündig mit Breitenangabe

\begin{document}
\noindent
\begin{center}
    \huge \textbf{Anlage II zur FKGO -- Kriterienkatalog}
\end{center}



\section{Höchstsätze}
Folgende Posten können über BFsG beantragt werden:\\
\begin{center}
\setlength\extrarowheight{2mm}
\sffamily
\begin{tabular}{l|L{4,5 cm}|L{5cm}|L{4,5cm}}
    &\textbf{Titel} & \textbf{Anmerkungen} & \textbf{Höchstsatz}\\ \hline
     a&Erstsemesterarbeit & in der ersten Woche des Semesters \newline im Sommersemester ist Vorhandensein von Erstsemestern nachzuweisen& 800 € \newline davon für Verpflegung max. 200 €\\ \hline
     b&Erstsemesterfahrten & max. 30 \% Nicht-Erstsemester \newline im Sommersemester ist Vorhandensein von Erstsemestern nachzuweisen & bis zu 30 Teilnehmer: 800 € \newline
     bis zu 50 Teilnehmer: 900 € \newline über 50 Teilnehmer: 1000 € \\ \hline
     c&Klausurfahrten & Fahrt für aktive Fachschaftsmitglieder, um gezielt an fachschafts- bezogenen Themen zu arbeiten \newline regional (< 100 km Anreise) & bis zu 10 Teilnehmer: 500 € \newline
     bis zu 30 Teilnehmer: 800 € \newline über 30 Teilnehmer: 900 € \\ \hline
     d&Teilnahme BuFaTa & Teilnahme landes-, bundes-, europa- oder weltweiten Fachschaftsversammlungen & 800 € \\ \hline
     e&Bildungsfahrt & mehrtägige Exkursion mit Fachbezug, offen für die gesamte Fachschaft & bis zu 20 Teilnehmer: 800 € \newline
     bis zu 50 Teilnehmer: 900 € \newline über 50 Teilnehmer: 1000 € \\ \hline
     f&Tagesexkursion & eintägige Exkursion mit Fachbezug, offen für die gesamte Fachschaft & bis zu 20 Teilnehmer: 100 € \newline
     bis zu 50 Teilnehmer: 250 € \newline über 50 Teilnehmer: 500 € \\ \hline
     g&Computer und Zubehör & Reparatur oder Kauf von EDV-Geräten, kein alltäglicher Bürobedarf wie Druckerpatronen & 400 € \\ \hline
     h&Inhaltliche Veranstaltungen & to do & 700 € \\ \hline
     i&Ausrichtung BuFaTa & überregionale Bundesfachschaftentagung in Bonn & 2000 € \\ \hline
     j&Fachschaftsneugründung & Maßgeblich ist die Aufnahme in die Fachschaftenliste (Anhang III FKGO) & AFsG-Sockelsatzes, i.d.R.1000 € \\
\end{tabular}
\end{center}
\pagebreak

\rmfamily
\section{Maximale Ausgaben}
\begin{enumerate}[(1)]
    \item Für Fahrten nach § 1 Punkt b, c, e, und f können je Person und Tag maximal 50 € beantragt werden.
    \item Für Fahrten nach § 1 Punkt b, c, e, und f können pro Semester maximal 2500 € beantragt werden.
    \item Für inhaltliche Veranstaltungen nach Punkt h können pro Semester maximal 2000 € beantragt werden.
\end{enumerate}
\end{document}
